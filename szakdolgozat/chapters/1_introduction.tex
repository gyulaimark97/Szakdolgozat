\Chapter{Bevezetés}

A szakdolgozat témája a tanult
numerikus számítási ejárások Python környezetben való bemutatása. Ilyen
numerikus problémák például a mátrix transzponálás mátrix felbontások
(LU, Cholesky), lineáris egyenetrendszerek megoldása, interpolációk vagy a numerikus deriválás és integrálás. A numerikus
eljárások mindig az egzakt megoldást vagy annak egy közelítését fogják adni.
Ilyen problémákra már sokféle szoftvert fejlesztettek ki. Például a
Matlab, de akár az R környezetet is használhatjuk ilyen célokra. De
akkor miért Python? -- teheti fel a kérdést a kedves Olvasó. A válaszom
pedig az, hogy a dolgozatomban azt fogom bemutatni, hogy, mint sokminden
máshoz ehhez is tökéletesen megfelel és könnyen használható ez a nyelv.

Az első fejezetben magáról a Python nyelvről fogok írni általánosan.
Bemutatom, hogy miről is szól a szakdolgozat, mi a témája, mit használtam fel benne,
mire is van szükség ha ki szeretné próbálni a benne található kódokat.
