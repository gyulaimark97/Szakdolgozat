\Chapter{Telepítés}

Ha szeretnénk Python-ban programozni ahoz szükségünk van egy
szövegszerkesztőre vagy integrált fejlesztő környezetre a Python
értelmezőn túl. A python értelmező ingyenesen elérhető a python
weboldaláról. A népszerűbb linux disztribúciók, mint az Ubuntu, Debian,
Linux Mint ...stb, már magába foglalja az értelmezőt ezért telepíteni
sem kell. A csomagok kezeléséhez szükségünk van a \texttt{pip} nevű
csomagkezelőre is. A \texttt{pip} segítségével tudjuk telepíten
lokálisan vagy globálisan a felhasználni kívánt csomagokat is melyek
előre megírt program könyvtárak tulajdonképpen, melyeket fel
használhatunk ahoz, hogy ne mindent nekünk kelljen implementálni. Ilyen
csomagok a Numpy, Matplotlib, Scipy, vagy a Jupyter notebook is melyeket
használok a későbbiekben.

    \subsection{Python telepítése}\label{python-telepuxedtuxe9se}

    Első lépésként le kell töltenünk a python értelmezőt a megfelelő
operációs rendszerünkre. Windows alatt a telpítőbe konfigurálhatjuk mit
telepítsünk, és azt is, hogy csak a saját felhasználónknak vagy pedig
mindenkinek szeretnénk telepíteni, útóbbihoz rendszergazdai jogosultság
szükséges.

Unix-szerű rendszereken sem sokkal nehezebb a dolgunk bár operációs
rendszerenként és disztribúciónként eltérések mutatkoznak.

A forráskódból való telepítéshez először is le kell tölteni a forrást,
szintén a python weboldaláról, majd kicsomagolni ezek után elnavigálunk
egy terminálban a mappába melybe kicsomagoltunk és futtatnunk kell a
configure nevezetű bash scriptet (\textbf{./configure} paranccsal tudjuk
megtenni). Miután végzett kiadjuk a \textbf{make} parancsot mellyel
elkezdődik a fordítási folyamat és felépül az alkalmazás esetünkben a
CPython értelmező. A következő lépés a \textbf{make test} mely
ellenőrzi, hogy minden rendben ment, majd jöhet a \textbf{sudo make
install} mellyel telepítjük az értelmezőt (make install előtt ott van a
sudo így láthatjuk hogy ehez szükségünk lesz root jogra (superuser)).

\emph{Megjegyzés:} - \emph{Itt ha a fordítás vagy a telepítés során
hibába ütközünk valószínüleg hiányzik valamilyen csomag a
számítógépünkről amit telepítenünk kell, ha ez megtörtént, újra le kell
futattni a parancsot amiben a hibát kaptuk és folytatódhat a folyamat
tovább.}

\begin{itemize}
\item
  \emph{Egyes linux disztribúciók rendelkeznek előre telepített
  értelmezővel, így felesleges telepíteni azt, ellenőrizzük le először,
  hogy telepítve van-e. Illetve némely linux disztribúció (főleg a
  debian alapúak mint, Debian, Ubuntu, Linux mint) különbséget tesz a
  python2 és python3 közözótt előbbit terminálból \textbf{python} míg
  utóbbit \textbf{python3} paranccsal érjük el és ez igaz a pip
  csomagkezelőre is, tehát, ha python3-al dolgozunk akkor a
  \textbf{pip3}-al tudunk hozzá csomagokat telepíteni.}
\end{itemize}

    A \texttt{pip} az új Python változatokban már a Python részeként
telepítésre kerül.

    \subsection{pip}\label{pip}

    A pipet egyszerűen lehet használni terminálban vagy windwos
parancssorban begépeljük hogy pip (vagy pip3) és megadjuk a csomag nevét
amit szeretnénk telepíteni: - \emph{pip somePackage} - \emph{python pip
-m somePackage}

Esetleg csinálhatunk olyat is, hogy összeírjuk a csomagokat egy
szövegfájlba (soronként egy csomagot) és telepítjük a - \emph{pip -r
csomagnevek.txt}

parancs segítségével.

pip weboldala: https://pip.pypa.io/en/stable/

    \subsection{Anaconda}\label{anaconda}

    A telepítés egy egyszerűsített változata, ha telepítjük az Anaconda
környezetet. Az Anaconda egy tudományos, gépi tanulásos platform,
melyben összegyűjtöttek több, mint 7500 python és R csomagot, melyeket
egyszerűen grafikus kezelő felület mellett telepíthetünk,ezen kívül
biztosít még IDE-ket (Spyder, Visual Studio Code) illetve a Jupyter
notebookot melyben ez a dolgozat is írodott. Az Anaconda telepítése
egyszerű. Felmegyünk a weboldalára, letöltjük és telepítjük Windows
alatt mind a 32 és 64 bites verzió támogatott míg Mac OS alatt
választhatunk grafikus vagy parancssoros telepítő között, Linux alatt
pedig támogatott az IBM Power 8 és 9 architektúrája is, az x86-64
mellett.

    \subsection{Jupyter munkafüzet}\label{jupyter-munkafuxfczet}

    A Jupyter notebookról még csak említés szintjén volt szó, most nézzünk
bele kicsit mélyebben mi is ez. A Jupyter notebook tulajdonképpen egy
nyílt forráskódú web alkalmazás mely lehetőséget nyújt arra, hogy olyan
interaktív dokumentumokat készítsünk és osszunk meg, melyek egyszerre
tartalmaznak futtatható kódokat és magyarázó szöveget, vizualizációkat.
A legelterjedtebben Python nyelvvel használják de támogat több, mint 40
programozási nyelvet köztük a C++, R, Ruby, julia és Calysto scheme
programozási nyelveket és a notebook ideális adat vizualizációra,
numerikus számításokhoz, gépi tanuláshoz és statisztikai modellekhez.

Maga a munkafüzet egy json szintaktikáju .ipynb kiterjesztésű fájl mely
úgynevezett cellákból áll. A celláknak vannak típusai: -
\textbf{markdown:} a markdown cellákban lehet a szöveget írni különböző
formázási lehetőségekkel. A markdown cellák támogatják a HTML elemeket
és a Latex matetmatikai módját, illetve más leíró nylevekből is vett át
egy két dolgot. - \textbf{code:} a code cellákban helyezhetjük el a
kódunkat melyet le szeretnénk futtatni - \textbf{heading:}
tulajdonképpen egy markdown cella csak rak a jupíter egy \#-ot az
elejére ami a legnagyobb headinget jelöli - \textbf{raw:} egy cella mely
nem kerül formázásra

A munkafüzet az IPython shellt használja, amely lehetővé teszi nekünk
ezt és az interaktív widget-ek használatát is a munkafüzetben, melynek a
segítségével így akár a kódunk paramétereit is megváltoztathatjuk.

