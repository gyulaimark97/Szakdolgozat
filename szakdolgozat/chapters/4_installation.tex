\Chapter{A Python környezet összeállítása}

Ha szeretnénk Python-ban programozni, ahhoz szükségünk van egy
szövegszerkesztőre vagy integrált fejlesztő környezetre a Python
értelmezőn túl. A python értelmező ingyenesen elérhető a Python
weboldaláról. A népszerűbb GNU/Linux disztribúciók (mint például a Debian, Ubuntu, Linux Mint) már beépítve tartalmazzák az értelmezőt, ezért azt telepíteni
sem kell. A csomagok kezeléséhez szükségünk van a \texttt{pip} nevű
csomagkezelőre is. A \texttt{pip} segítségével tudjuk telepíten
lokálisan vagy globálisan a felhasználni kívánt csomagokat is, melyek
előre megírt program könyvtárak tulajdonképpen, melyeket fel
használhatunk ahhoz, hogy ne mindent nekünk kelljen implementálni. Ilyen
csomagok a Numpy, Matplotlib, Scipy, vagy a Jupyter notebook is melyeket
használok a későbbiekben.

\Section{Python telepítése}

    Első lépésként le kell töltenünk a Python értelmezőt a megfelelő
operációs rendszerünkre. Windows alatt a telpítőbe konfigurálhatjuk mit
telepítsünk, és azt is, hogy csak a saját felhasználónknak vagy pedig
mindenkinek szeretnénk telepíteni, útóbbihoz rendszergazdai jogosultság
szükséges.

Unix-szerű rendszereken sem sokkal nehezebb a dolgunk, bár operációs
rendszerenként és disztribúciónként eltérések mutatkoznak.

A forráskódból való telepítéshez először is le kell tölteni a forrást,
szintén a Python weboldaláról, majd kicsomagolni, ezek után elnavigálunk
egy terminálban a mappába melybe kicsomagoltunk és futtatnunk kell a
configure nevezetű bash scriptet (\texttt{./configure} paranccsal tudjuk
megtenni). Miután végzett, kiadjuk a \texttt{make} parancsot mellyel
elkezdődik a fordítási folyamat és felépül az alkalmazás, esetünkben a
CPython értelmező. A következő lépés a \texttt{make test} mely
ellenőrzi, hogy minden rendben ment, majd jöhet a \texttt{sudo make
install} mellyel telepítjük az értelmezőt (\texttt{make install} előtt ott van a
\texttt{sudo} így láthatjuk hogy ehez szükségünk lesz root jogra (superuser)).
(Itt ha a fordítás vagy a telepítés során
hibába ütközünk valószínüleg hiányzik valamilyen csomag a
számítógépünkről amit telepítenünk kell, ha ez megtörtént, újra le kell
futattni a parancsot amiben a hibát kaptuk és folytatódhat a folyamat
tovább.)
Néhány további észrevétel az értelmezővel és a verziójával kapcsolatban az alábbi pontokban olvasható.
\begin{itemize}
\item
  Egyes Linux disztribúciók rendelkeznek előre telepített
  értelmezővel, így felesleges telepíteni azt, ellenőrizzük le először,
  hogy telepítve van-e. Illetve némely Linux disztribúció (főleg a
  Debian alapúak, mint az Ubuntu vagy a Linux mint) különbséget tesz a
  \texttt{python2} és \texttt{python3} közözótt. Előbbit terminálból \texttt{python} míg
  utóbbit \texttt{python3} paranccsal érjük el.
\item Ez igaz a \texttt{pip}
  csomagkezelőre is, tehát, ha \texttt{python3}-al dolgozunk akkor a
  \texttt{pip3}-al tudunk hozzá csomagokat telepíteni.
\item
A \texttt{pip} az új Python változatokban már a Python részeként
telepítésre kerül.
\end{itemize}

\SubSection{pip}

A \texttt{pip}-et egyszerűen lehet használni terminálban vagy ha a Windows
parancssorban begépeljük hogy \texttt{pip} (vagy \texttt{pip3}) és megadjuk a csomag nevét
amit szeretnénk telepíteni:
\begin{python}
pip somePackage
python pip -m somePackage
\end{python}
Esetleg csinálhatunk olyat is, hogy összeírjuk a csomagokat egy
szövegfájlba (soronként egy csomagot) és telepítjük a - 
\begin{python}
pip -r csomagnevek.txt
\end{python}
parancs segítségével.
TODO: Hivatkozás! % pip weboldala: https://pip.pypa.io/en/stable/

\SubSection{Anaconda}

    A telepítés egy egyszerűsített változata, ha telepítjük az Anaconda
környezetet. Az Anaconda egy tudományos, gépi tanulásos platform,
melyben összegyűjtöttek több, mint 7500 Python és R csomagot, melyeket
egyszerűen, grafikus kezelő felület mellett telepíthetünk, ezen kívül
biztosít még IDE-ket (Spyder, Visual Studio Code) illetve a Jupyter
notebookot, melyben a dolgozathoz készített példák is íródtak. Az Anaconda telepítése
egyszerű. Felmegyünk a weboldalára, letöltjük és telepítjük.
Windows
alatt mind a 32, mind pedig a 64 bites verzió támogatott míg Mac OS alatt
választhatunk grafikus vagy parancssoros telepítő között.
Linux alatt támogatott az IBM Power 8 és 9 architektúrája is, az x86-64
mellett.

\SubSection{Jupyter munkafüzet}

A Jupyter notebookról még csak említés szintjén volt szó az eddigiekben. Most nézzünk
bele kicsit mélyebben mi is ez. A Jupyter notebook tulajdonképpen egy
nyílt forráskódú web alkalmazás mely lehetőséget nyújt arra, hogy olyan
interaktív dokumentumokat készítsünk és osszunk meg, melyek egyszerre
tartalmaznak futtatható kódokat és magyarázó szöveget, vizualizációkat.
A legelterjedtebben Python nyelvvel használják de támogat több, mint 40
programozási nyelvet köztük a C++, R, Ruby, julia és Calysto scheme
programozási nyelveket.
A notebook ideális adat vizualizációra,
numerikus számításokhoz, gépi tanuláshoz és statisztikai modellekhez.

Maga a munkafüzet egy JSON szintaktikájú \texttt{.ipynb} kiterjesztésű fájl, mely
úgynevezett cellákból áll. A celláknak az alábbi típusai vannak.
\begin{itemize}
\item \texttt{markdown}: a markdown cellákban lehet a szöveget írni különböző
formázási lehetőségekkel. A markdown cellák támogatják a HTML elemeket
és a \LaTeX\ matematikai módját, illetve más leíró nyelvekből is vett át
elemeket.
\item \texttt{code}: ezekben a cellákban helyezhetjük el a
kódunkat, melyet le szeretnénk futtatni.
\item \texttt{heading}:
Tulajdonképpen egy MarkDown cella, csak rak a Jupyter egy \#-ot az
elejére, ami a legnagyobb headinget jelöli.
\item \texttt{raw}: egy cella mely
nem kerül formázásra.
\end{itemize}
A munkafüzet az IPython shellt használja, amely lehetővé teszi nekünk
ezt, és az interaktív widget-ek használatát is a munkafüzetben, melynek a
segítségével így akár a kódunk paramétereit is megváltoztathatjuk.
