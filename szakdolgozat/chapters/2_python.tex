\Chapter{Python}

\section{Bevezető}\label{bevezetux151}

    Ebben a rövid fejezetben szeretném a kedves Olvasónak reprezentálni
miről is szól a szakdolgozat, mi a témája, mit használtam fel benne,
mire is van szükség ha ki szeretné próbálni a benne található kódokat.
Vágjunk is bele! Először is a téma: a szakdolgozat témája a tanult
numerikus számítási ejárások Python környezetben való bemutatása. Ilyen
numerikus problémák például a mátrix transzponálás mátrix felbontások
(LU, Cholesky), lineáris egyenetrendszerek megoldása, interpolációk
felírása, vagy a numerikus deriválás és integrálás. A numerikus
eljárások mindig az egzakt megoldáshoz közelítő értéket fognak megadni.
Ilyen problémákra már sokféle szoftvert fejlesztettek ki. Például a
Matlab, de akár az R környezetet is használhatjuk ilyen célokra. De
akkor miért Python? - teheti fel a kérdést a kedves Olvasó. A válaszom
pedig az, hogy a dolgozatomban azt fogom bemutatni, hogy, mint sokminden
máshoz ehhez is tökéletesen megfelel és könnyen használható ez a nyelv.
A következő szakaszban magáról a Python nyelvről fogok írni általánosan,
majd áttekintem a dolgozat fő részeit.

    \subsubsection{Mi is az a Python?}\label{mi-is-az-a-python}

    A Python egy nagyon magas szintű, platform független, általános
programozási nyelv szóval itt nem a kígyófélék egy családjára vagy Monty
Pythonra (bár a nevét a a Monty Python csapatról kapta) kell gondolni.
Egy könnyen elsajátítható programozási nyelv mely jelen pillanatban a 3
legnépszerűbb között van a világon (2020 Február). A nyelv interpretált
és támogatja a objektum orientált, a funkcionális, az imperatív és a
procedurális programozási paradigmákat, valamint a dinamikus típusokat
és dinamikus memóriakezelést (használ \emph{garbage collector}t). A
nyelvet Guido van Rossum holland programozó kezdte el fejleszteni
1989-ben, majd nyilvánosságra hozta 1991-ben. Ez volt a 0.9-es
verziószámú. 1994-ben megjelent az 1.0-ás majd 2000-ben a 2.0-ás verzió
és csak 2008-ban követte a dolgozatban általam is használt Python 3. A a
3-as és a 2-es nem minden esetben kompatibilis egymással és a 2-es
utolsó támogatott verzióinak (2.7.x) a támogatása is megszűnt 2020
januárban. A nyelvet napjainkban már a PSF (\emph{Python Software
Foundation}) fejleszti és kezeli.

    \subsubsection{Felhasználása különböző
területeken}\label{felhasznuxe1luxe1sa-kuxfcluxf6nbuxf6zux151-teruxfcleteken}

    A Python nyelvet sok területen felhasználják, többek között

\begin{itemize}
\item
  Webfejlesztésben,
\item
  Tudományos és numerikus számításoknál,
\item
  Oktatásban,
\item
  Asztali grafikus felhasználói felűletek (GUI) fejlesztésében,
\item
  Szoftverfejlesztésben,
\item
  Üzleti szoftverek fejlesztésében.
\end{itemize}

Látható tehát hogy tényleg sok helyen jól használható ez a nyelv.

    \subsubsection{Mi szükséges a
hozzá?}\label{mi-szuxfcksuxe9ges-a-hozzuxe1}

    Tulajdonképpen csak egy számítógép, a Python interpreter és egy
támogatott operációs rendszer, amit nem lesz nehéz találni hiszen a
nyelv minden ma használt és elterjedt operációs rendszert támogat. Az
interpreter pedig letölthető a python weboldaláról (www.python.org),
illetve egyes operációs rendszerekben alapértelmezés szerint telepítve
van (különböző linux disztribúciók).

    \subsubsection{Milyen problémákkal fog foglalkozni a
dolgozat?}\label{milyen-probluxe9muxe1kkal-fog-foglalkozni-a-dolgozat}

    Ahogy már fent említettem, a dolgozat Python-ban fogja bemutatni a
tanult numerikus problémákat és módszereket, melyekkel egy közelítő
megoldást adhatunk egy-egy problémára. Ilyen prolémák:

\begin{itemize}
\item
  Hibaszámítás: Abszulút és relatív hiba
\item
  Vektor és mátrix műveletek
\item
  Lineáris egyenletrendszerek megoldása
\item
  Interpolációk
\item
  Numerikus deriválás
\item
  Numerikus Integrálás
\item
  Sajátérték és sajátvektor
\end{itemize}

Ezeken belül is az ilyen módszerekkel fogok dolgozni, mint például:

\begin{itemize}
\item
  Gauss módszer
\item
  Gauss-Jordan módszer
\item
  Legkisebb négyzetek módszere
\item
  Lagrange interpolációk
\item
  Spline interpolációk
\item
  Téglalap módszer
\item
  Trapéz módszer
\end{itemize}


