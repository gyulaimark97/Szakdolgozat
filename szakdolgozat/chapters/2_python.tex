\Chapter{Python}

\Section{A Python programozási nyelv}

A Python egy nagyon magas szintű, platform független, általános célú
programozási nyelv, szóval itt nem a kígyófélék egy családjára vagy Monty
Pythonra (bár a nevét a a Monty Python csapatról kapta) kell gondolni.
Egy könnyen elsajátítható programozási nyelv mely jelen pillanatban a 3
legnépszerűbb között van a világon (2020 Február).
TODO: Hivatkozás!

A nyelv interpretált
és támogatja a objektum orientált, a funkcionális, az imperatív és a
procedurális programozási paradigmákat, valamint a dinamikus típusokat
és dinamikus memóriakezelést, szemétgyűjtő algoritmust (használ \emph{garbage collector}t). A
nyelvet Guido van Rossum holland programozó kezdte el fejleszteni
1989-ben, majd nyilvánosságra hozta 1991-ben. Ez volt a 0.9-es
verziószámú. 1994-ben megjelent az 1.0-ás majd 2000-ben a 2.0-ás verzió
és csak 2008-ban követte a dolgozatban általam is használt Python 3. A a
3-as és a 2-es nem minden esetben kompatibilis egymással és a 2-es
utolsó támogatott verzióinak (2.7.x) a támogatása is megszűnt 2020
januárban. A nyelvet napjainkban már a PSF (\emph{Python Software
Foundation}) fejleszti és kezeli.

\Section{Felhasználása különböző
területeken}

A Python nyelvet sok területen felhasználják, többek között
\begin{itemize}
\item
  Webfejlesztésben,
\item
  Tudományos és numerikus számításoknál,
\item
  Oktatásban,
\item
  Asztali grafikus felhasználói felűletek (GUI) fejlesztésében,
\item
  Szoftverfejlesztésben,
\item
  Üzleti szoftverek fejlesztésében.
\end{itemize}
Látható tehát hogy tényleg sok helyen jól használható ez a nyelv.

\Section{Hardveres és szoftveres követelmények}

Tulajdonképpen csak egy számítógép, a Python interpreter és egy
támogatott operációs rendszer, amit nem lesz nehéz találni hiszen a
nyelv minden ma használt és elterjedt operációs rendszert támogat. Az
interpreter pedig letölthető a python weboldaláról (www.python.org),
illetve egyes operációs rendszerekben alapértelmezés szerint telepítve
van (különböző linux disztribúciók).

\Section{A dolgozatban vizsgált problémák}

Ahogy már fent említettem, a dolgozat Python-ban fogja bemutatni a
tanult numerikus problémákat és módszereket, melyekkel egy közelítő
megoldást adhatunk egy-egy problémára. Ilyen prolémák:
\begin{itemize}
\item
  Hibaszámítás: Abszulút és relatív hiba,
\item
  Vektor és mátrix műveletek,
\item
  Lineáris egyenletrendszerek megoldása,
\item
  Interpolációk,
\item
  Numerikus deriválás,
\item
  Numerikus Integrálás,
\item
  Sajátérték és sajátvektor.
\end{itemize}
Ezeken belül is az ilyen módszerekkel fogok dolgozni, mint például:
\begin{itemize}
\item
  Gauss módszer,
\item
  Gauss-Jordan módszer,
\item
  Legkisebb négyzetek módszere,
\item
  Lagrange interpolációk,
\item
  Spline interpolációk,
\item
  Téglalap módszer,
\item
  Trapéz módszer.
\end{itemize}
