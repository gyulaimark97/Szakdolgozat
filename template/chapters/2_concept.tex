\Chapter{Szoftvereszközök numerikus számításokhoz}

\Section{Alapvető elvárások}

TODO: Röviden el kell mondani, hogy milyen elvárások vannak a szoftverekkel szemben.
- Lehessen programokat írni benne. (Nem feltétlenül kell, hogy általános célú programozási nyelv legyen.)
- Lehessen a kapott eredményeket grafikus formában megjeleníteni.
- Lehessen dokumentálni az egyes részlépéseket.
- Lehessen adatokat fájlból olvasni, fájlba írni. I/O támogatás. Formátumok (pl. CSV)
- Számítási teljesítmény (inkább szimulációk esetén, oktatásnál annyira nem szempont).
- Legyen elterjedt, tehát a hallgatók később ipari szinten is tudják alkalmazni az ismereteket.
- Ne legyen túlságosan nehezen tanulható. (Például procedurális paradigmára vagy OOP elvekre épüljön.)

\Section{Elérhető szoftverek}

TODO: Érdemes szót ejteni arról, hogy
- mennyire elterjedtek,
- elsődlegesen milyen feladatra készültek,
- hogy férhetők hozzá (fizetős, ingyenes, nyílt forráskódú),
- milyen platformokon működnek.

Ehhez a fejezet végére egy összesítő táblázat is jól jöhet. 

TODO: Mindegyiknél érdemes hangsúlyozni pro- és kontra a jellemzőket.

TODO: Mindegyikhez érdemes hivatkozásokat is rakni.

\SubSection{MATLAB}

\SubSection{Python/NumPy}

A Python az egyik legnépszerűbb programozási nyelv.
TIOBE index.

\SubSection{GNU R}

Statisztikai területen nagyon elterjedt.
Nyílt forráskódú.

\SubSection{Maple}

\SubSection{Wolfram: Mathematica, Wolfram Alpha}

Nem szükséges hozzá külön telepítés.
A levezetések és teljes ábrák fizetősek.

\SubSection{Fortran}

Régi, implementációs limitek (pl. sorok hossza)

\Section{Python eszközkészlet}

TODO: Áttekintő jelleggel írni a
- Python-ról, nyelvről,
- értelmezőjéről (IPython-ról külön) és
- fejlesztőeszközökről (Spyder és Jupyter), PyCharm,
- grafikus megjelenítéshez Matplotlib,
- NumPy.
